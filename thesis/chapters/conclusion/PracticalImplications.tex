\section{Practical Implications}
We believe that the findings of this thesis provide valuable insights for organizations and individuals considering the adoption of microfrontend architecture in their projects. Our analysis of various implementation approaches and their evaluation aims to assist decision-makers in selecting the most suitable approach for their specific use cases.

Using the chosen approach, we successfully developed a fully functional prototypical microfrontend application that demonstrated key advantages in terms of reusability and extensibility, as outlined in Chapter \ref{chap:EvaluationResults}. The implementation of this application validated the feasibility of employing a Web Components-based approach within an enterprise environment, showcasing its effectiveness in addressing common microfrontend challenges such as communication, composition, routing, and styling. Given these results, we consider this approach a viable and scalable option for real-world applications.