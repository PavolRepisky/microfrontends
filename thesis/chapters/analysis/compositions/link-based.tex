\subsection{Link-based composition}
One of the simplest approaches to a microfrontend architecture is the composition of multiple applications using hyperlinks. In this method, each such application is developed by a dedicated team that creates its own HTML, CSS, and JavaScript files. Each application is deployed independently, typically on a different port under the same domain, ensuring it remains fully isolated from other applications in the system. To facilitate navigation between applications, teams must share their URL patterns that will be used to link their sites from other teams' applications.\cite{Geers} Example of such URL patterns might look like the following.
\begin{itemize}
\item $\textbf{Team Project - Project page}$ \newline
URL pattern: \url{http://localhost:3000/projects/<project-id>}
\item $\textbf{Team Task - Task page}$ \newline
URL pattern: \url{http://localhost:3001/tasks/<task-id>}
\end{itemize}

\noindent
Rather than exchanging these URL patterns verbally, which would require informing all affected teams and potentially redeploying their applications when URLs change, a better solution is to maintain a centralized JSON file containing all URL patterns for each team, which can then other applications read at runtime to construct the necessary URLs. To compose the microfrontends, we simply interconnect them across different applications using standard anchor elements. As an example, let's consider an online store application that consists of a catalog microfrontend and product microfrontends. The link from the catalog to a specific product would look as follows.
\begin{verbatim}
<!-- Using an anchor tag to link to a product within the catalog page 
-->
<a href="http://localhost:3001/products/123">View Product #123</a>
\end{verbatim}

\subsubsection{Advantages}
This composition method offers several benefits. It provides complete isolation between microfrontends, eliminating direct communication between them. Errors in one microfrontend are contained and do not affect other parts of the system. The microfrontends can be deployed independently, and the project requires minimal configuration. The system can be easily expanded by adding new microfrontends \cite{Geers}.

\subsubsection{Disadvantages}
The primary limitation of this approach is that it does not allow for the integration of multiple microfrontends on a single page. Users must click through links and wait for page loads when navigating between different sections, potentially worsening the user experience.  Additionally, common elements, such as headers, need to be reimplemented and  maintained separately within each microfrontend, leading to code duplication, maintenance overhead, and potential inconsistencies. Resource sharing between such applications is also very complex. \cite{Geers}

\subsubsection{Suitability}
While microfrontend composition via links provides a basic integration strategy, it is rarely used as the sole approach in modern web development due to its limitations in resource sharing between microfrontends. It is typically combined with other techniques to create more robust solutions. \cite{Geers}

\begin{table}[h]
   \centering
   \begin{tabular}{|l|l|}
      \hline
         \textbf{Aspect} & \textbf{Score} \\
      \hline
         Extensibility & 4/5 \\
      \hline
         Reusability & 2/5 \\
      \hline
         Simplicity & 5/5 \\
      \hline
         Performance & 3/5 \\
      \hline
         Resource sharing & 1/5 \\
      \hline
         Developer experience & 4/5 \\
      \hline
   \end{tabular}
   \caption{Characteristics assessment of link-based composition}
   \label{table:links-composition}
\end{table}