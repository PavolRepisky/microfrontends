\section{Results}
The purpose of this thesis was to review the current literature on microfrontends architecture and explore various approaches to designing and implementing it. The goal was to compare these approaches in aspects such as reusability, extensibility, resource sharing, and application state management. Furthermore, the thesis aimed to identify the approaches best suited for enterprise application development and to design and implement a prototypical microfrontend application using one of these approaches. \\

\noindent
Based on the reviewed literature, we conducted a comprehensive analysis of the microfrontends architecture, its history, challenges, advantages, disadvantages, practical use cases, and implementation appraoches. We compared seven of the most frequently mentioned approaches in the aspects mentoined above, going a step further by listing their advantages and disadvantages and defining the environments in which they perform best. Among these seven approaches, we identified three that are most suitable for enterprise environments and selected a Web Components-based approach for the implementation of the prototypical application, with an explanation behind the decision. \\

\noindent
We selected the application purpose, defined its requirements, designed its microfrontends-based system architecture, and analyzed various technologies best suited for our use case. From this analysis, we chose a set of technologies for implementation and designed the application's wireframes. Based on this design, we developed the prototypical application in Angular 18 using Web Components. Common microfrontends challenges such as communication, composition, routing, and styling were addressed, with solutions provided in this thesis. The resulting application was evaluated in terms of reusability, extensibility, resource sharing, and application state management. The complete code is freely available in our GitHub repository: \url{https://github.com/PavolRepisky/microfrontends}. 