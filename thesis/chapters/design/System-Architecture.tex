\section{System Architecture}
In this section, we will delve into the system architecture of the application, which adopts a microfrontend architecture to separate different domains into distinct, independently developed, and deployed microfrontends. The application is entirely frontend-focused, built using Angular, and follows a modular design to ensure scalability, maintainability, and flexibility. This section will discuss the role of the application shell, the individual microfrontends, and how they interact with each other. Each of these modules will be managed by dedicated imaginary teams to reflect the independence and separation of concerns inherent in microfrontend architectures.

\subsection{Application Shell}
The application shell is the core of the application, functioning as the orchestrator for all microfrontends. It provides critical infrastructure and functionality, including routing, internationalization (i18n), and data communication. This shell is responsible for rndering idividual microfrontends, based on the route the user navigates to. It uses Angular Router to determine which microfrontend should be displayed at any given time, and loads them as standalone JavaScript bundles fetched from separate servers. \\\\
The shell maintains the master routing configuration of the entire application. Each microfrontend has its own routing module, but the shell takes care of routing at the highest level, enabling deep linking and lazy loading of microfrontends as needed. And it passes essential data to the microfrontends, such as: current route or current language. Additionaly the shell listens for and dispatches Browser Events to facilitate communication between microfrontends. The application shell will be deployed as a separate module.

\subsection{Users Microfrontend}
This microfrontend is responsible for user management functionality within the system. It is designed as a standalone module that can be deployed independently, and its key functionality revolves around managing users through CRUD operations (Create, Read, Update, Delete). The users are displayed in a list view for easy management.

\subsection{Projects Microfrontend}
The Projects Microfrontend handles project management, providing functionality to create, update, view, and delete projects. Like the Users module, the projects are presented in a list view format. Additionally, this microfrontend is responsible for assigning users to projects.

\subsection{Tasks Microfrontend}
The Tasks Microfrontend offers task management features, utilizing a Kanban board to display and organize tasks. This microfrontend enables the creation, updating, and deletion of tasks, and supports assigning users to tasks as well as associating tasks with specific projects. The Kanban board provides a visual representation of task progression, allowing users to move tasks between different stages (e.g., "Backlog", "In Progress", "Completed").

\subsection{Dashboard Microfrontend}
The Dashboard Microfrontend serves as the homepage of the application, providing an overview of relevant statistics and key metrics across users, projects, and tasks. It presents a consolidated view, offering users a high-level summary of the application’s current state. The dashboard aggregates data from other microfrontends to display statistics such as the number of active projects, tasks in progress, and users. It serves as a passive consumer of data, with minimal interaction beyond displaying information to the user.