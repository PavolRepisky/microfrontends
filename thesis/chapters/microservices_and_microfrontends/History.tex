\section{History of Microservices and Microfrontends}
By the beginning of this millennium, some software systems had become so large that they were difficult to maintain. The most famous example is Amazon's \cite{Amazon} e-shop, which went into critical condition in 2002. Hundreds of developers worked on the system. Although the system was divided into layers and components, these components were tightly interconnected. The development of a new version of the system was slow, and deployment took on the order of weeks. Amazon then came up with a key solution: splitting the monolithic application into separately deployable services and creating an automated pipeline from the build to the deployment of the application \cite{Brigham}.

Amazon's example was later followed by other companies such as Netflix \cite{Netflix}, Spotify \cite{Spotify}, Uber \cite{Uber}, and others \cite{Kwiecień}. In 2011, the first stand-alone workshop on this new approach to architecture was held near Venice. The workshop called this new architectural style {Microservices} \cite{Fowler}. In 2014, James Lewis and Martin Fowler wrote a blog in which they generalized the ideas from the workshop. In the article, they defined what Microservices are and specified their basic characteristics \cite{Fowler}. The article popularized Microservices in the general professional community. In 2015, Sam Newman wrote the first book on microservices, \emph{Building Microservices} \cite{NewmanBuildingMs}. In parallel, Chris Richardson created his website \cite{RichardsonWeb} in 2014. He later compiled the ideas from this site into a separate book, \emph{Microservices Patterns} \cite{RichardsonPatterns}.

The term \emph{Microfrontends} first appeared in the \emph{ThoughtWorks Technology Radar} magazine \cite{ThoughtWorksRadar} at the end of 2016. Through six subsequent editions, it climbed from the "assess" and "trial" sections to the "adopt" section. They described it as the application of microservices concepts to the frontend. Ideas from this magazine were further developed by Cam Jackson in an article \emph{Micro Frontends} published on the Martin Fowler website. Another major milestone was the creation of a website on the topic of micro frontends by Michael Geers \cite{GeersWeb} in 2017. Later, in 2021, he compiled this site into a comprehensive monograph, \emph{Micro Frontends in Action} \cite{Geers}. In 2021, a second monograph by Luca Mezzalira, \emph{Building Micro-Frontends}, was published, further exploring this architecture. From the history, we see that Microfrontends are following a very similar path to microservices.