\section{Routing}
Routing is primarily handled in the application shell, for the most part using the standard approach via @angular/router. When a route for a component inside the application shell is requested, we simply load the component in the standard way. If a route for a microfrontend is requested, we first check if the microfrontend can be loaded via a route guard. If it can, we then lazy load the microfrontend by appending a script that points to the correct URL. This script subsequently renders the microfrontend. \\

\noindent
If a microfrontend has multiple routes of its own, this requires specific handling. In such cases, a route change listener in the application shell will detect route changes. When a route is changed, the application shell will pass an updated attribute for the route to the microfrontend, causing it to re-render. The microfrontend will have its own router, which uses the route information from the application shell to load the correct page. \\

\noindent
The microfrontend itself must also listen for its own route changes. If the route is changed within the microfrontend, it will inform the application shell via CustomEvents. The application shell will then handle the routing accordingly and pass the new route to the microfrontend. 