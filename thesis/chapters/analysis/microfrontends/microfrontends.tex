\section{Microfrontends in detail}
\subsection{Overview}
Microfrontends is an architectural and organizational approach to building web applications, essentially extending the concepts of Microservices to the frontend. The main idea behind it is to break down a monolithic codebase into multiple small, isolated, and loosely coupled applications, which are then combined into one. Each application can be developed, run, and deployed independently, focusing on a single feature or business sub-domain. \\

\noindent
In traditional web applications, teams are typically split based on the technologies they use, such as frontend and backend teams. However, microfrontends aim to divide the application vertically, meaning the teams are cross-functional, with members from all departments. These teams develop the application from its presentation layer to its data layer and are fully responsible for the feature. Teams can choose the technologies they want to use based on their preferences and project requirements. Typically, each microfrontend would be an autonomous application with its own dedicated backend and continuous delivery pipeline. Each microfrontend can be built with completely different implementation techniques, even though they together compose a single web application. \\

\noindent
A microfrontend can either be an entire page or just a "fragment"—an element that typically appears on multiple pages, such as the header or footer. The process of putting all the microfrontends together is called composition or integration. There are several ways to do this, and a comparison of various approaches will be presented in the next section. These approaches are separated into two categories: server-side and client-side, based on where the composition occurs. Other concepts that form the foundation of microfrontends are:
\begin{itemize}
    \item \textbf{Code isolation:} The runtime should not be shared, and applications should be self-contained.
    \item \textbf{Native browser API preference:} For communication purposes, the native browser’s APIs should be preferred. 
    \item \textbf{Focus on automation:} A strong focus is placed on automation to speed up the development process.
    \item \textbf{Hiding implementation details:} Contracts should be defined upfront between teams to avoid reliance on implementation details that may change.
    \item \textbf{Decentralized governance:} Decisions are moved away from a one-size-fits-all approach and into the hands of the teams.
    approach
\end{itemize}

\subsection{Benefits}
There are numerous reasons why large companies are adopting microfrontends architecture, but in most cases, the number one reason is to increase development speed and decrease time to market. The reason why development in microfrontends architecture is much faster is due to cross-functional teams. All the people involved in creating a feature are in the same team, resulting in significantly less communication overhead, without unnecessary waiting for responses. Additionaly, due to their different perspectives, they come up with more creative and effective solutions for the tasks. \\\\
\noindent
Microfrontends also bring most of the benefits of microservices architecture to the frontend side. The application is split into smaller pieces (microfrontends), resulting in a smaller, more understandable, and less complex codebase with shorter onboarding times. Each microfrontend is isolated, therefore independently deployable, and its failure does not affect the rest of the system. The implementation decisions are decentralized to the teams working on the features, giving them a stronger sense of ownership of the given feature. \\\\
\noindent
Microfrontends encourage changes and experimentation with new technologies, which is very important, especially in the frontend space, where technologies change rapidly. What was considered state-of-the-art in frontend development last year might now be deprecated. Microfrontends provide a way to upgrade only small parts one by one instead of rewriting the entire application at once. It is also easier to migrate old applications since the architecture allows for a new application to run side by side with the old one. Lastly, teams deliver features directly to the customer instead of relying on an additional team, increasing customer focus.  \cite{Geers}\cite{Montelius}
 
\subsection{Drawbacks}
However, everything comes with a cost. Having an application split into multiple parts across teams, potentially using different technologies, naturally introduces a lot of code redundancy. This can lead to larger file sizes and, consequently, longer download times, so performance should be closely monitored for this reason. Each team also needs to set up and maintain its own application server, build process, and continuous integration pipelines. \\\\
\noindent
A bug in a library used across multiple microfrontends must be fixed in each one. If not managed properly, common libraries may be shipped in multiple microfrontends, further increasing the bundle size. This architecture also complicates routing and communication within the system, requiring them to be handled in a specialized manner. \\\\
\noindent
Lastly, it introduces new challenges, such as orchestrating the microfrontends (e.g., determining when each microfrontend should be displayed) and maintaining UI/UX consistency across the entire system. To address the second one, a design system should be developed.\cite{Geers}\cite{Pavlenko}

\subsection{Use cases}
Microfrontends make scaling projects easier and development faster. However this may not apply, for small projects with just a few people, where communication is not a significant issue, in this case microfrontends may introduce more drawbacks than benefits. Large, complex projects with many people involved gain the most advantages from this architecture. Below are examples of projects that have adopted this architecture.  \\

\noindent
The first noteworthy example is \textbf{Amazon}. Several employees have reported that the e-commerce site has been using this architecture for quite some time. It uses a UI integration technique to assemble parts of the page before it is displayed. Other e-commerce companies, such as the well-known \textbf{IKEA} and the European fashion e-commerce platform \textbf{Zalando}, have also adopted microfrontends. Zalando even open-sourced their microfrontends framework Tailor.js, later replacing it with the Interface framework. Another prominent adopter is \textbf{Spotify}. Their desktop application initially used iframes to compose the UI, but this approach was abandoned due to poor performance and was replaced by a SPA. Similarly, \textbf{SAP} also utilized iframes and released their own microfrontends framework, designed for creating enterprise applications that integrate with SAP systems. Lastly, \textbf{DAZN}, a sports streaming platform, migrated its monolithic frontend to a microfrontends architecture. DAZN focused on supporting not only the web but also multiple smart TVs and gaming consoles.\cite{Geers} \cite{MezzaliraBuildingMf}

