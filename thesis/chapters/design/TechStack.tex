\section{Tech Stack}
This section outlines the technologies used in the implementation of the prototypical microfrontend application.

\subsubsection*{Angular}
The primary framework used for application development is Angular 18, the most recent version available at the time of writing. Angular is a robust, platform-agnostic web development framework developed by Google, designed for building scalable, maintainable, and high-performance single-page applications (SPAs). In this project, Angular is used to develop both the Application Shell and the individual microfrontends. The choice of Angular is motivated by its strong adoption in enterprise environments and its built-in support for Web Components.

\subsubsection*{TypeScript}
TypeScript serves as the main programming language for the application. As the standard language for Angular development, TypeScript extends JavaScript by introducing static typing, which helps catch errors during compile time rather than at runtime. This enhances reliability, reduces bugs, and improves code maintainability. Its rich tooling ecosystem and type safety make it particularly well-suited for enterprise-level applications, further justifying its selection for this project.

\subsubsection*{Web Components}
To ensure that each microfrontend operates independently and can be seamlessly integrated into the Application Shell, Web Components are employed. In this implementation, each microfrontend is exposed as a custom element. The Shadow DOM is used to encapsulate styles and avoid CSS conflicts, thereby enforcing strict isolation and reusability across microfrontends.

\subsubsection*{Custom Events}
Communication between the various microfrontends is facilitated using browser-native \texttt{CustomEvent} objects. This event-driven approach enables decoupled microfrontends to share data and notify each other of interactions (such as selections) without creating tight dependencies. It ensures a flexible, scalable, and modular system architecture.

\subsubsection*{Node.js}
The backend of the application is implemented using Node.js, a lightweight, event-driven JavaScript runtime environment optimized for scalable network applications. To streamline backend development, Node.js is used in conjunction with Express.js—a minimal and flexible framework that simplifies the creation of RESTful APIs. This backend setup enables rapid prototyping and allows the development effort to remain focused on the frontend aspects of the application.
