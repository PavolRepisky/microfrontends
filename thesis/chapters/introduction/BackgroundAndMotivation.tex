\section{Background and Motivation}
Microfrontends are one of the most promising directions in frontend development of modern software applications. They are based on the same idea as the more well-known microservices: splitting a software application into several separate smaller, independent applications to better address issues such as complexity, maintainability and deployability. \cite{Mezzalira}\cite{Geers}\cite{Pavlenko}\\

\noindent
While microservices focus on the partitioning the application layer, microfrontends deal with the partitioning of the frontend (presentation) layer of an application. Thus, despite some similarities, microservices and microfrontends address a different set of problems and use different techniques and technologies. Another important difference is that there are several accepted principles and patterns in microservices \cite{Richardson}\cite{Newman}. In microfrontends, there is not yet a general consensus on which of the principles is most appropriate, nor is there a set of design patterns for microfrontends. \\

\noindent
Several articles \cite{Pavlenko}\cite{Peltonen}\cite{Montelius}\cite{Jackson} have been written on different approaches to microfrontends. However, each of these articles explains the approaches only briefly and does not analyze them in depth nor compare them with each other. An overview work on the basic approaches to microfrontends has been provided by Geers \cite{Geers} in his book \emph{Micro Frontends in Action}. This book not only summarized the basic approaches to creating microfrontends and provided simple examples, but also compare them at a general level. On the other hand, the examples given in the book are too simple for enterprise applications, and even the comparison of approaches is left at a rather general level. The same can be said about the book \emph{Building Micro-Frontends} by Mezzalira \cite{Mezzalira}.