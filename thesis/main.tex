% \documentclass[12pt, twoside]{book}
\documentclass[12pt, oneside]{book}  % jednostranna tlac

%spravne nastavenie okrajov
\usepackage[a4paper,top=2.5cm,bottom=2.5cm,left=3.5cm,right=2cm]{geometry}
%zapnutie fontov pre UTF8 kodovanie
\usepackage[utf8]{inputenc}
\usepackage[T1]{fontenc}
\usepackage{amsmath}
\usepackage{enumitem}
\usepackage{array}

%zapnutie slovenskeho delenia slov
%a automatickych nadpisov ako Obsah, Obrázok a pod. v slovencine
%\usepackage[slovak]{babel} % vypnite pre prace v anglictine!

%nastavenie riadkovania podla smernice
\linespread{1.25} % hodnota 1.25 by mala zodpovedat 1.5 riadkovaniu

% balicek na vkladanie zdrojoveho kodu
\usepackage{listings}
% ukazky kodu su cislovane ako Listing 1,2,...
% tu je Listing zmenene na Algoritmus 1,2,...
\renewcommand{\lstlistingname}{Algorithm}
% nastavenia balicka listings
% mozete pridat aj language=...
% na nastavenie najcastejsie pouzivaneho prog. jazyka
% takisto sa da zapnut cislovanie riadkov
\lstset{frame=lines}

% balicek na vkladanie obrazkov
\usepackage{graphicx}
% balicek na vkladanie celych pdf dokumentov, tu zadanie
\usepackage{pdfpages}
% balicek na spravne formatovanie URL
\usepackage{url}
% balicek na hyperlinky v ramci dokumentu
% zrusime farebne ramiky okolo liniek aby pdf
% vyzeralo rovnako ako tlacena verzia
\usepackage[hidelinks,breaklinks]{hyperref}

% -------------------
% --- Definicia zakladnych pojmov
% --- Vyplnte podla vasho zadania, rok ma byt rok odovzdania
% -------------------
\def\mfrok{2024}
\def\mftitle{Analysis, Design and Implementation of Micro-frontend Architecture}
\def\mftyp{Bachelor Thesis}
\def\mfauthor{Bc. Pavol Repiský}
\def\mfskolitel{RNDr. Ľubor Šešera, PhD.}

%ak mate konzultanta, odkomentujte aj jeho meno na titulnom liste
\def\mfkonzultant{Ing. Juraj Marák}  

\def\mfmiestocas{Bratislava, \mfrok}
\def\mfuniverzita{COMENIUS UNIVERSITY IN BRATISLAVA}
\def\mffakulta{FACULTY OF MATHEMATICS PHYSICS AND INFORMATICS}
\def\mftypprace{Diploma thesis}

\def\mfodbor{Computer Science}
\def\program{Applied Computer Science}

% Ak je školiteľ z FMFI, uvádzate katedru školiteľa, zrejme by mala byť aj na zadaní z AIS2
\def\mfpracovisko{Department of Computer Science}

\begin{document}     
\frontmatter
\pagestyle{empty}

\noindent
\begin{minipage}{\textwidth}
    \begin{center}
      \textbf{\mfuniverzita\\
      \mffakulta}
    \end{center}
\end{minipage}

\vfill
\begin{figure}[!hbt]
	\begin{center}
		\includegraphics[width=0.4\textwidth]{images/FMFI_logo_BP.png}
		\label{img:logo}
	\end{center}
\end{figure}
\begin{center}
		\textbf{\MakeUppercase{\Large\mftitle}}\\
    \mftypprace
\end{center}
\vfill
\mfrok \hfill
\mfauthor
%\eject 
\cleardoublepage
% --- koniec obalky ----



% -------------------
% --- Titulný list
% -------------------
\thispagestyle{empty}
\noindent
\begin{minipage}{\textwidth}
    \begin{center}
      \textbf{\mfuniverzita\\
      \mffakulta}
    \end{center}
\end{minipage}

\vfill
\begin{figure}[!hbt]
    \begin{center}
        \includegraphics[width=0.4\textwidth]{images/FMFI_logo_BP.png}
        \label{img:logo_dark}
    \end{center}
\end{figure}

\begin{center}
	\textbf{\MakeUppercase{\Large\mftitle}}\\
	\mftypprace
\end{center}
\vfill


\begin{tabular}{l l}
Study program: & \program \\
Branch of study: & \mfodbor \\
Department: & \mfpracovisko \\
Supervisor: & \mfskolitel \\
Consultant: & \mfkonzultant \\
\end{tabular}

\vfill
\noindent
\mfmiestocas \hfill
\mfauthor
%\eject 
\cleardoublepage
% --- Koniec titulnej strany



% -------------------
% --- Zadanie z AIS
% -------------------
% v tlačenej verzii s podpismi zainteresovaných osôb.
% v elektronickej verzii sa zverejňuje zadanie bez podpisov
% v pracach v anglictine anglicke aj slovenske zadanie

\newpage
\setcounter{page}{2}
\includepdf{images/zadanie.pdf}
\includepdf{images/zadanie-en.pdf}
% --- Koniec zadania


% -------------------
%   Poďakovanie - nepovinné
% -------------------
\newpage
\thispagestyle{empty}
\chapter*{Acknowledgement}\label{chap:thank_you}
I would like to thank RNDr. Ľubor Šešera, PhD., for his supervision, willingness, and valuable advice. I am also very thankful to Ing. Juraj Marák for his patience, guidance, and all the time he dedicated to me during the preparation of my thesis.

\vfill\eject 
% --- Koniec poďakovania

% -------------------
%   Abstrakt - Slovensky
% -------------------
\newpage 
\thispagestyle{empty}
\chapter*{Abstrakt}\label{chap:abstract_sk}
Táto práca podrobne analyzuje niekoľko najpoužívanejších prístupov k mikrofrontendom a porovnáva ich na základe aspektov, ako sú rozšíriteľnosť, opätovná použiteľnosť, jednoduchosť, výkon, zdieľanie zdrojov a skúsenosti vývojárov. Každý z týchto prístupov je dôkladne posúdený z hľadiska jeho výhod, nevýhod a použiteľnosti. Jeden z týchto prístupov, konkrétne web-compoents, bol zvolený na implementáciu proof-of-concept aplikácie pre manažovanie projektov vo frameworku Angular. Počas jej vývoju sme narazili na niekoľko bežných problémov, ako sú kompozícia, zdieľanie zdrojov, smerovanie a izolácia, ktoré sme vyriešili a riešenia uvádzame priamo v práci. \\

\noindent
Výsledky implementácie naznačujú, že webové komponenty sú v efektívnym a dobrým prístupom k implementácii mikrofrontendovej architektúry. Prinášajú však niekoľko výziev, z ktorých hlavnými sú riešenie spoločných závislostí a komunikácia medzi mikroaplikáciami. Úspešné adoptovanie tejto architektúry si preto vyžaduje dôkladné plánovanie. Zistenia tejto práce poskytujú cenné poznatky o týchto výzvach a ich riešeniach. Okrem toho zjednodušujú rozhodovanie vývojárom a organizáciám, ktoré zvažujú túto architektúru pre svoje projekty.

\paragraph*{Kľúčové slová:} Microfrontends, Web-components, architektúry webových aplikácií, Angular
% --- Koniec Abstrakt - Slovensky


% -------------------
% --- Abstrakt - Anglicky 
% -------------------
\newpage 
\thispagestyle{empty}
\chapter*{Abstract}\label{chap:abstract_en}
This paper analyzes in detail several of the most widely used approaches to micro-frontends and compares them based on aspects such as extensibility, reusability, simplicity, performance, resource sharing, and developer experience. Each of these approaches is thoroughly assessed in terms of its advantages, disadvantages and usability. One of these approaches, namely web-compoints, was chosen to implement a proof-of-concept project management application in the Angular framework. During its development, we encountered several common issues such as composition, resource sharing, routing, and isolation, which we solved and will present solutions directly in the thesis. \\

\noindent
The implementation results suggest that web components are an efficient and good approach to implement a micro-frontend architecture. However, they present several challenges, the main ones being the sharing of common dependencies and communication between micro-applications. Successful adoption of this architecture therefore requires careful planning. The findings of this work provide valuable insights into these challenges and their solutions. In addition, they simplify decision making for developers and organizations considering this architecture for their projects.

\paragraph*{Keywords:} Microfrontends, Web-components, web application architectures, Angular
% --- Koniec Abstrakt - Anglicky

% -------------------
% --- Predhovor - v informatike sa zvacsa nepouziva
% -------------------
%\newpage 
%
%
%\chapter*{Preface} %
%
%Predhovor je všeobecná informácia o práci, obsahuje hlavnú charakteristiku práce 
%a okolnosti jej vzniku. Autor zdôvodní výber témy, stručne informuje o cieľoch 
%a význame práce, spomenie domáci a zahraničný kontext, komu je práca určená, 
%použité metódy, stav poznania; autor stručne charakterizuje svoj prístup a svoje
%hľadisko. 
%
% --- Koniec Predhovor


% -------------------
% --- Obsah
% -------------------

\newpage 

\tableofcontents

% ---  Koniec Obsahu

% -------------------
% --- Zoznamy tabuliek, obrázkov - nepovinne
% -------------------

\newpage 

\listoffigures
\listoftables

% ---  Koniec Zoznamov

\mainmatter
\pagestyle{headings}

\input chapters/introduction/Introduction.tex 
\input chapters/microservices_and_microfrontends/MicroservicesAndMicrofrontends.tex
\input chapters/technological_foundations/TechnologicalFoundations.tex
\input chapters/microfrontends_in_detail/MicrofrontendsInDetail.tex
\input chapters/design/Design.tex
\input chapters/implementation/Implementation.tex
\input chapters/Conclusion.tex

%\input zaver.tex

% -------------------
% --- Bibliografia
% -------------------


\newpage	

\backmatter

\thispagestyle{empty}
\clearpage

\bibliographystyle{ieeetr}
\bibliography{literatura} 

%Prípadne môžete napísať literatúru priamo tu
%\begin{thebibliography}{5}
 
%\bibitem{br1} MOLINA H. G. - ULLMAN J. D. - WIDOM J., 2002, Database Systems, Upper Saddle River : Prentice-Hall, 2002, 1119 s., Pearson International edition, 0-13-098043-9

%\bibitem{br2} MOLINA H. G. - ULLMAN J. D. - WIDOM J., 2000 , Databasse System implementation, New Jersey : Prentice-Hall, 2000, 653s., ???

%\bibitem{br3} ULLMAN J. D. - WIDOM J., 1997, A First Course in Database Systems, New Jersey : Prentice-Hall, 1997, 470s., 

%\bibitem{br4} PREFUSE, 2007, The Prefuse visualization toolkit,  [online] Dostupné na internete: <http://prefuse.org/>

%\bibitem{br5} PREFUSE Forum, Sourceforge - Prefuse Forum,  [online] Dostupné na internete: <http://sourceforge.net/projects/prefuse/>

%\end{thebibliography}

%---koniec Referencii

% -------------------
%--- Prilohy---
% -------------------

%Nepovinná časť prílohy obsahuje materiály, ktoré neboli zaradené priamo  do textu. Každá príloha sa začína na novej strane.
%Zoznam príloh je súčasťou obsahu.
%
%\input appendixA.tex

%\input appendixB.tex

\end{document}






