\section{Introduction to Microfrontends}  
The main idea behind microfrontends is to extend the principles of microservices to the frontend side \cite{Montelius, Geers}. In microfrontends, the presentation layer is split into smaller, more manageable pieces \cite{Montelius}. Jackson \cite{Jackson} describes microfrontends as: ``An architectural style where independently deliverable frontend applications are composed into a greater whole.'' Each microfrontend can be developed, tested, and deployed independently while still appearing to customers as a single cohesive product \cite{Jackson}.

As Geers \cite{Geers} describes, microfrontends are not just an architectural approach but also an organizational one. Microfrontends can be combined with microservices on the backend, resulting in a system divided into several full-stack micro-applications \cite{Montelius}. Each such micro-application is autonomous, with its own continuous delivery pipeline and serving a specific business domain or feature \cite{Peltonen}. Microfrontends introduce vertical teams instead of horizontal ones. Instead of being grouped by the development technologies they use, teams are grouped by application features \cite{Montelius}. Each team is responsible for a specific business feature or requirement, which they implement end-to-end. The team takes complete ownership of the feature, decentralizing decision-making and determining what technology to use. Figure \ref{fig:microfrontends-architecture}, originally presented by Geers \cite{GeersWeb}, shows a high-level diagram illustrating how an application can be divided into vertical microapplications. Each microapplication is managed by a separate team, responsible for everything from the presentation layer to the persistence layer, with each team focused on a specific business objective or mission.
\begin{figure}[h]  
  \centerline{\includegraphics[width=1\textwidth]{images/microfrontends-architecture.png}}  
  \caption[Microfrontend Architecture]{High-level architecture of a microfrontend-based application, source \cite{GeersWeb}.}  
  \label{fig:microfrontends-architecture}  
\end{figure}