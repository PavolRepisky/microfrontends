\section{Extensibility}
Extensibility refers to the ability of a system to be enhanced with new functionality or modifications without requiring significant changes to existing components. Our microfrontend architecture is designed with extensibility in mind.

In our application, extensibility is achieved mainly through microfrontend architecture that supports both internal evolution and external expansion. New features can be added, or existing ones modified, with minimal impact on the rest of the system.

This extensibility takes several forms. First, changes within a single microfrontend are well-isolated due to their strong encapsulation. Internal modifications—such as updating business logic or UI elements—do not affect other parts of the application. Even interface-level changes, such as introducing new input properties or emitting additional events, remain manageable as long as communication contracts are respected. These changes typically require only the addition of new listeners in the corresponding microfrontends or the passing of additional attributes from the application shell.

Second, our architecture is designed to make the addition of entirely new features straightforward. New functionality is typically introduced as a separate microfrontend, allowing teams to develop and deploy independently. Since microfrontends are standard Angular applications that are wrapped into custom elements only at build time, development remains familiar and efficient. New microfrontends can be easily added to the Application Shell by declaring their routes and defining their host components. One of the technical achievements of our approach is the ability to run multiple Angular applications concurrently within the same browser environment—something that was traditionally considered infeasible.

However, extensibility comes with trade-offs. The most critical issue is application size. Each microfrontend includes its own dependencies, which leads to linear growth in the total bundle size. For example, the bundled size of the User Microfrontend is about 1.21MB, consisting of:
\begin{itemize}
    \item \texttt{styles.css} - 272KB (22.5\%) - primarily Bootstrap styles,
    \item \texttt{polyfills.js} - 35KB (2.9\%) - for legacy browser support,
    \item \texttt{main.js} - 904KB (74.6\%) - actual application logic.
\end{itemize}
Similarly, the Task Microfrontend is about 1.25MB in total, with:
\begin{itemize}
    \item \texttt{styles.css} - 260KB (20.9\%),
    \item \texttt{polyfills.js} - 35KB (2.8\%),
    \item \texttt{main.js} - 952KB (76.3\%).
\end{itemize}

On average, a microfrontend in our application contributes approximately 1.23MB. With this size, when using the Angular framework, it would be possible to build a real-world application consisting of roughly 15 to 30 microfrontends. The theoretical number could be even higher, since Chromium-based browsers have a per-tab limit of up to 500MB, which—given the size of our microfrontends—translates to around 400 microfrontends. The real limitation, however, would be performance. Based on our experience, at around 20-30MB total (which corresponds to 15-30 of our microfrontends), we have not observed a significant drop in performance yet. However, the initial load time would already be quite long, and it would require the use of optimization techniques such as lazy loading and asset sharing.