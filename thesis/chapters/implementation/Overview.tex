\section{Overview}
The application was developed using Angular 18 and is composed of three separate Angular projects. The first project is the \texttt{application-shell}, which serves as the container for all microfrontends. The second project, \texttt{user-management}, represents the user microfrontend, and the third project, \texttt{task-management}, represents the task microfrontend.

For practical purposes, all these projects were generated within the same Angular workspace. This approach was chosen due to the absence of distinct teams working on the individual projects. In a real-world scenario, each microfrontend would be developed as a separate Angular project to facilitate independent development and deployment. The workspace was created using the standard Angular CLI command as follows.
\begin{lstlisting}
  ng new project-management-tool --create-application=false
\end{lstlisting}
Afterward, the application-shell project was generated using the standard method.
\begin{lstlisting}
  ng generate application application-shell
\end{lstlisting}
After generating the application, Bootstrap and \texttt{@ng-bootstrap/ng-bootstrap} were installed and configured in the standard way, along with \texttt{@ngx-translate}, a library for internationalization support. Interfaces were then designed using placeholder components for the microfrontends. Support for theme switching (light/dark) and language switching (Slovak/English) was added, with preferences saved in local storage.

Microfrontends were generated in the same way, but with the use of the \texttt{--prefix} option, as follows.
\begin{lstlisting}
  ng generate application user-management --prefix=user
\end{lstlisting}
The development of microfrontends followed a similar approach to the Application Shell, with one key difference: the way the application is bootstrapped. Depending on the environment, the bootstrapping process varies. The microfrontends were then built and statically served using the \texttt{http-server} package, each on a different port via the command line:
\begin{lstlisting}
  npx http-server dist/user-management/browser -p 4201
\end{lstlisting}
To avoid repeatedly typing these commands, they were added as script commands in \texttt{package.json}. The composition and loading of microfrontends will be further explained in the next section.

The microfrontends communicate with a backend, which is a standard Node.js server using Express.js. Each microfrontend has been assigned its own URL prefix, such as \texttt{/users} for the User Microfrontend and \texttt{/tasks} for the Task Microfrontend. The microfrontends interact with the shared backend using Angular's \texttt{HttpClient} for CRUD operations. We are not using a database; instead, data is stored only at runtime. In a real-world application, each microfrontend that requires a backend would ideally have its own microservice and dedicated database. However, due to the scope of this thesis and its focus on the frontend aspects of microfrontends, we chose a simplified approach to avoid unnecessary complexity.