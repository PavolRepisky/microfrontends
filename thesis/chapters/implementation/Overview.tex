\section{Overview}
The application was developed using Angular 18 and consists of three separate Angular projects. The first one is the \texttt{application-shell}, which acts as a container for all microfrontends. The second one is \texttt{user-management}, representing the user microfrontend, and the last one is \texttt{task-management}, representing the task microfrontend.\\

\noindent
All these projects were generated under the same Angular workspace. This was done purely for practical reasons, as there are no distinct teams working on the projects. In a real-world application, each microfrontend should be developed as a separate Angular project. The workspace was generated in the standard way using the following command.
\begin{verbatim}
  ng new project-management-tool --create-application=false
\end{verbatim}
Afterward, the application-shell project was generated using the standard method.
\begin{verbatim}
  ng generate application application-shell
\end{verbatim}
After generating the application, Bootstrap and \texttt{@ng-bootstrap/ng-bootstrap} were installed and configured in the standard way, along with \texttt{@ngx-translate}, a library for internationalization support. Interfaces were then designed using placeholder components for microfrontends. Support for theme switching (light/dark) and language switching (Slovak/English) was added, with preferences saved in local storage. The \texttt{application-shell} was developed as a standard Angular application. \\

\noindent
Microfrontends were generated in the same way but with the use of the --prefix option, such as following.
\begin{verbatim}
  ng generate application user-management --prefix=user
\end{verbatim}
The development of microfrontends followed a similar approach to the \texttt{application-shell}, with one key difference: the way the application is bootstrapped. Depending on the environment, the bootstrapping process varies. The microfrontends were then built and statically served using \text{http-server} package, each on a different port via the command line:
\begin{verbatim}
  npx http-server dist/user-management/browser -p 4201
\end{verbatim}
To avoid repeatedly typing these commands, they were added as script commands in \texttt{package.json}. The composition and loading of microfrontends will be further explained in the next section. \\

\noindent
The microfrontends communicate with a backend, which is a standard Node.js server with Express.js. Each microfrontend has been assigned its own URL prefix, such as \texttt{/users} for the user microfrontend and \texttt{/tasks} for the task microfrontend. The microfrontends interact with the backend using Angular \texttt{HttpClient} for CRUD operations. We are not using any database; instead, data is stored only at runtime. In a real-world application, each microfrontend requiring a backend should ideally have its own microservice and dedicated database. However, due to the scope of this thesis and the focus on the frontend aspects of microfrontends, we chose a simplified approach to avoid unnecessary complexity.