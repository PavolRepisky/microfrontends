\section{Application Requirements}
This section presents the functional and non-functional requirements for the application, categorized by priority as follows:
\begin{itemize}
   \item (M) Must: Crucial for the system's operation.
   \item (S) Should: Highly recommended, though not mandatory.
   \item (C) Could: Optional for implementation.
\end{itemize}

\subsection{Functional Requirements}
This subsection provides a comprehensive list of all functional requirements for the application.

\subsubsection*{User Management}
For simplicity, user registration and login will not be implemented. User roles will serve informational purposes only, as every user will have full access to all functionalities.
\begin{itemize}
   \item Users can create, update, and delete other users (M).
   \item Users are displayed in a table view (M).
   \item Users can set and update the following attributes: name, email, phone number, role, status, and bio (M).
   \item Users can be filtered by role (S).
\end{itemize}

\subsubsection*{Task Management}
A similar approach applies to tasks: all users will be able to manage any task.
\begin{itemize}
   \item Users can create, update, and delete tasks (M).
   \item Tasks are displayed on a Kanban board (M).
   \item Users can set and update the following task attributes: title, status, priority, tag, due date, and description (M).
   \item Tasks can be assigned to users (M).
\end{itemize}

\subsubsection*{Dashboard}
\begin{itemize}
   \item A new users widget is displayed on the dashboard (M).
   \item A new tasks widget is displayed on the dashboard (M).
   \item The widgets must communicate with each other (M).
   \item The dashboard displays a monthly user count graph (C).
   \item The dashboard displays a monthly task count graph (C).
\end{itemize}

\subsubsection*{Settings}
\begin{itemize}
   \item Users can switch between light and dark themes (C).
   \item Users can switch between Slovak and English languages (C).
\end{itemize}

\subsection{Non-functional Requirements}
This subsection presents a comprehensive list of all non-functional requirements for the application.
\begin{itemize}
   \item The application is divided into several microfrontends (M).
   \item Each microfrontend is isolated to prevent cascading failures (M).
   \item The application can be easily scaled by adding new microfrontends (M).
   \item The application is intuitive and easy to use (S).
   \item The microfrontends are easily customizable (S).
\end{itemize}