\chapter{Theoretical Framework}
\label{chap:Theoretical-Framework} 

\section{Architecture Types}
Explore various architectural patterns prevalent in web development, including monolithic, microservices, serverless, and progressive web app (PWA) architectures. Analyze the characteristics and suitability of each type, providing a comparative study to guide web developers in choosing the most appropriate architecture for their specific project requirements.

\section{Communication Protocols}
Investigate communication mechanisms and protocols essential for seamless interactions between web components. Discuss common approaches like RESTful APIs, GraphQL, and WebSocket protocols. Evaluate the trade-offs associated with each protocol, considering factors such as performance, scalability, and real-time capabilities, offering insights to help web developers make informed communication strategy decisions.

\section{Composition Strategies}
Investigate communication mechanisms and protocols essential for seamless interactions between web components. Discuss common approaches like RESTful APIs, GraphQL, and WebSocket protocols. Evaluate the trade-offs associated with each protocol, considering factors such as performance, scalability, and real-time capabilities, offering insights to help web developers make informed communication strategy decisions.

\section{State Management}
Investigate the nuances of managing state within web applications. Cover aspects like client-side and server-side state management, data synchronization, and state persistence. Discuss the impact on user experience, data consistency, and scalability, and evaluate popular state management libraries and patterns within the context of web development.

\section{Versioning Strategies}
Explore strategies for versioning web applications to ensure smooth deployment, updates, and maintenance. Discuss versioning techniques such as Semantic Versioning, API versioning, and content delivery network (CDN) caching. Provide guidance on selecting appropriate versioning strategies based on the unique needs and constraints of web development projects.
