\section{Web Application Architectures}
\subsection{Server-side Rendered Application}
Server-side rendered applications are web applications in which the majority of the HTML content is generated on the server before it is sent to the user's browser. Which can also include fetching data from APIs, composing components together, and applying styling. This is in contrast to client-side rendered applications, where the browser generates the HTML content after receiving data from the server.\\\\
Server-side rendering offers improved performance by delegating rendering to the server. It also provides easy indexation by search engines and better accessibility. A server-side rendered application speeds up initial page load time. However, there are also some trade-offs to consider when using SSR. Applications may experience higher server load since HTML must be rendered on the server for each request, and therefore, higher cost. Managing application state can be more complex, and a number of third-party libraries and tools are incompatible with SSR. Additionally, SSR may cause slower page rendering in case of frequent server requests.\\\\
Server-side rendering is ideal for static HTML site generation, which does not require too many requests, or for projects with a focus on SEO and accessibility, such as blogging platforms.