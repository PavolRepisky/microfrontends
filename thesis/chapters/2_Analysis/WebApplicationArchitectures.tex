\section{Web Application Architectures}
\subsection{Monolithic Applications}
A monolithic application represents a traditional architectural style where all components are interconnected and tighly coupled, forming a single, unified codebase, that operates independently from other applications. It couples all of the business concerns together. When some changes are made to any part of the application, it necessitates rebuilding and redeploying the entire codebase. Development is typically horizontally partitioned into frontend, backend, and database layer teams, thereby decomposing the code into three distinct layers.  The monolith can be scalled up, by running multiple its instances behind a load-balancer. \\\\
Monolithic applications are easier to debugg, test, deploy and monitor, given that everything is encompassed within a single codebase. Using a singular API, as opposed to multiple ones, can potentially enhance performance, and the initial project configuration is small. However, this architecture comes with numerous disadvantages. The tightly coupled nature of large monolithic codebases makes them challenging to fully understand, resulting in more complex and slow development processes. Individual components cannot be scaled independently, and an error in one such component can affect the entire application. Moreover, transitioning to new technologies is nearly impracticable as it necessitates rewriting the entire application. \\\\
Thus, the monolithic approach is best suited for simple, small-scale applications that do not anticipate frequent code changes or evolving scalability requirements such as content-based websites.

\subsection{Server-side Rendered Application}
Server-side rendered applications are web applications in which the majority of the HTML content is generated on the server before it is sent to the user's browser. Which can also include fetching data from APIs, composing components together, and applying styling. This is in contrast to client-side rendered applications, where the browser generates the HTML content after receiving data from the server.\\\\
Server-side rendering offers improved performance by delegating rendering to the server. It also provides easy indexation by search engines and better accessibility. A server-side rendered application speeds up initial page load time. However, there are also some trade-offs to consider when using SSR. Applications may experience higher server load since HTML must be rendered on the server for each request, and therefore, higher cost. Managing application state can be more complex, and a number of third-party libraries and tools are incompatible with SSR. Additionally, SSR may cause slower page rendering in case of frequent server requests.\\\\
Server-side rendering is ideal for static HTML site generation, which does not require too many requests, or for projects with a focus on SEO and accessibility, such as blogging platforms.

\subsection{Single Page Applications}
Single page application or SPA for short, is a modern architecture approach to the development of client-side rendered web applications in which all essential resources are downloaded during the initial page load. As users interact with the application, the DOM undergoes dynamic updates via JavaScript and HTTP requests to fetch data from the server, eliminating the need for full page reloads.  Frameworks like React, Angular, and Vue have streamlined SPA development by offering robust tools for managing application state, routing, and data binding. \\\\
The key adavtanges of SPAs are a dynamic, app-like user experience, enhanced performance and reduced server load. The obvious downside of this architecture is a longer initial load time, which can be improved by lazy loading. Furthermore, the codebase may quiclky expand expand and become challenging to maintain. Multiple teams working on the same codebase, but different areas of the application can result in conflicts and communication overhead. \\\\
SPAs are best suited for highly responsive and interactive applications such as editors or design tools.

\subsection{Microservices}
Microservices is an architectural style for developing web applications as a suite of loosely coupled, independently deployable small services. Each service is responsible for a specific business concern and can be developed, deployed, and scaled independently of other services. Each of these services typically uses its own databases and communicate with other services using lightweight mechanisms, such as an HTTP resource API. They rely on automated processes mainly for deployment, and each service can be written in a different programming language. \\\\
Microservices offer enhanced team productivity by splitting into small, focused teams to concentrate on particular services and their development, deployment, and maintenance without being burdened by the complexities of the entire system. This also results in quicker deployment times. As the services are independent of each other, errors in one service does not affect entire system. The system can be easily scaled by adding new services or multiple instances of existing ones. They are technology-agnostic, and teams can experiment with new features and technologies, which can lead to innovations. Because of the smaller codebases, developers can better understand the functionality of particular services. On the other hand, microservices require a lot of complex upfront configuration. Since the services are independent, it's hard to manage communication between them, leading to difficult integration testing and debugging. As microservices communicate over a network, this can cause longer response times. Microservices require a high amount of automation and a complex infrastructure, which can lead to increased costs and operational overhead.\\\\
Microservices are best suited for building large and complex software applications with large teams, particularly those expected to expand, scale, and change.