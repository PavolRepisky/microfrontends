\section{Results}
The purpose of this thesis was to review the current literature on microfrontend architecture and explore various approaches to its design and implementation. The goal was to compare these approaches in terms of reusability, extensibility, resource sharing, and application state management. Furthermore, the thesis aimed to identify the approaches best suited for enterprise application development and to design and implement a prototypical microfrontend application using one of these approaches.

Based on the reviewed literature, we conducted a comprehensive analysis of microfrontend architecture—its challenges, advantages, disadvantages, practical use cases, and implementation approaches. We compared seven of the most frequently referenced approaches in the aspects of extensibility, reusability, simplicity, performance, resource sharing, and developer experience, listing their advantages and disadvantages and identifying the environments in which they perform best. Among these, we identified three approaches as most suitable for enterprise environments: server-side composition, composition via Module Federation, and composition using Web Components. We selected the Web Components-based approach for implementing the prototypical application and explained the reasoning behind this choice.

We defined the aim of our prototypical application, specified its requirements, designed its microfrontend-based system architecture, and analyzed various technologies best suited to our use case. Based on this analysis, we selected a set of implementation technologies and created the application's wireframes. Using this design, we developed the prototypical application in Angular 18 with Web Components. Common challenges in microfrontend development—such as communication, composition, routing, and styling—were addressed, and corresponding solutions are provided in this thesis. The resulting application was evaluated in terms of reusability, extensibility, resource sharing, and application state management. The complete source code is freely available in our GitHub repository: \url{https://github.com/PavolRepisky/microfrontends}.