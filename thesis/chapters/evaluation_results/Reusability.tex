\section{Reusability}
Reusability refers to the ability of software components — in our case, microfrontends — to be used across multiple applications or within different parts of the same application with minimal modifications.

In our resulting application, even though it is relatively simple and small, we achieved a high degree of reusability. Each microfrontend functions not only as a standalone page — such as for user or task management — that can be reused across other projects but also integrates within the same application as dashboard widgets.

Furthermore, our application leverages Web Components, which allow components to be reused in any environment or framework. Well-designed boundaries and domains for the microfrontends also enable their reuse across different applications. For example, the user microfrontend can be used in almost any application that supports user roles, as user management is an essential part of most systems.

However, our implementation also introduces a challenge to reusability, as microfrontends cannot be directly styled externally. This is due to two factors: first, they come with their own dependencies and do not share any with the parent application; second, they utilize the Shadow DOM, which isolates their DOM tree, making it inaccessible from the outside.
