\section{Practical Implications}
The findings of this thesis provide valuable insights for organizations and individuals considering the adoption of microfrontend architecture in their projects. Our detailed analysis of various implementation approaches and their evaluation aims to assist decision-makers in selecting the most suitable approach for their specific use cases. \\

\noindent
With our chosen approach, we successfully developed a fully functional microfrontend application that demonstrated key advantages in terms of reusability and extensibility as outlined in \ref{chap:EvaluationResults}. The implementation of the prototypical application validated the feasibility of utilizing a Web Components-based approach within an enterprise environment, showcasing its effectiveness in handling common microfrontend challenges such as communication, composition, routing, and styling. Given these results, we consider this approach a viable and scalable option for real-world applications.