\section{Requirements}
The requirements for the application are categorized by priority as follows:
\begin{itemize}
   \item (M) Must: Crucial for the system's operation.
   \item (S) Should: Highly recommended, though not mandatory.
   \item (C) Could: Optional for implementation.
\end{itemize}

\subsection{Functional Requirements}
Here is a comprehensive list of all functional requirements for the application.\\\\
\textbf{Project Management}
\begin{itemize}
   \item Users can create, update, and delete projects. (M)
   \item Projects are displayed in an easy-to-comprehend manner. (M)
   \item Users can plan projects by specifying start and end dates. (S)
   \item Project states can be set by users. (S)
   \item Projects can be viewed on a timeline. (C)
\end{itemize}
\  \\
\textbf{Task Management}
\begin{itemize}
   \item Users can create, update, and delete tasks. (M)
   \item The task listing page presents all tasks along with their respective information. (M)
   \item Users can link tasks to projects. (M)
   \item Task states can be set by users. (S)
   \item Tasks can be viewed on a kanban board. (C)
\end{itemize}
\  \\
\textbf{Collaboration Features}
\begin{itemize}
   \item Project owners can invite other users to join projects via email. (S)
   \item Project owners can remove project members. (S)
\end{itemize}
\  \\
\textbf{Dashboard}
\begin{itemize}
   \item A dashboard-like page contains information about new members, active tasks, and active projects. (M)
\end{itemize}

\subsection{Functional Requirements}
Here is a comprehensive list of all non-functional requirements for the application.
\begin{itemize}
   \item The application must be divided into several microfrontends. (M)
   \item Each microfrontendis isolated from others to prevent cascading failures. (M)
   \item The microfrontends comunicate via custom events. (M)
\end{itemize}