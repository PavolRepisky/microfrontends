\subsection{Microservices}
Microservices architecture can be defined as an approach to developing web applications, where each component runs in its own process and communicates through lightweight mechanisms, often an HTTP resource API. Each service is built around specific business capabilities and is independently developed, testable, deployable through fully automated deployment processes, and scalable. Each service can be written in a different programming language, has its own storage, and is managed by its own team. Usually, each service has a dedicated team, enabling the system to be developed in parallel. This approach places great emphasis on continuous delivery and deployment, service isolation, and decentralized decision-making. \\

\noindent
Microservices solve many problems associated with monolithic architectures. Due to the smaller size of services, they are less complex, easier to understand, and simpler to onboard new developers. Changes in one service do not affect the rest of the system, so only the modified service needs to be rebuilt and redeployed. Services are faster to reboot and start. Each service can be scaled independently based on its usage, rather than scaling the entire system. Teams can choose different technologies based on their preferences. Since each team is responsible for its own service, there is less communication overhead, and the codebase is more consistent. \\

\noindent
However, this approach also comes with certain challenges. Projects are much harder to initiate and require significant upfront configuration. Developers need a broader range of skills and knowledge. Since the services are independent, managing communication between them can be challenging, making end-to-end testing and debugging more difficult. Communication between services over a network can result in longer response times and potential delays. Additionally, microservices require complex infrastructure, which can lead to increased costs and operational overhead. \\

\noindent
Therefore, microservices are best suited for large and complex web applications with large teams, especially those expected to expand, scale, and undergo frequent changes. \cite{Fowler}\cite{Peltonen}