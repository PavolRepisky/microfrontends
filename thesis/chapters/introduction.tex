\chapter{Introduction}
\label{chap:Introduction} 

\section{Background and Motivation}
This section provides an insightful overview of the background that led to the initiation of the research, addressing the historical context and the driving forces that underscore the significance of the study. It aims to offer readers a clear understanding of the factors and events that motivated the exploration of micro-frontend architecture in web applications.

\section{Objectives and Scope}
Focused on defining the purpose and boundaries of the research, this section outlines the specific objectives that the study aims to achieve. It delineates the scope of the investigation, detailing the aspects, dimensions, or features of micro-frontend architecture that are within the purview of the research. This section serves as a roadmap, guiding readers on what to expect from the study.

\section{Key Research Questions}
Central to the research inquiry, this section formulates and presents the key research questions that guide the investigation. These questions encapsulate the core issues and uncertainties that the study seeks to address. They provide a framework for the subsequent analysis and evaluation of existing approaches to web application design using micro-frontend architecture.

\section{Scope and Limitations}
Acknowledging the constraints and boundaries of the study, this section transparently discusses the limitations and delimitations inherent in the research. It addresses potential challenges, external factors, and constraints that might influence the research outcomes. By clearly defining these limitations, readers gain a realistic understanding of the study's scope and potential implications.

