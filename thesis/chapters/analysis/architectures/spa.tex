\subsection{Single-page applications}
A Single-page Application (SPA) is a popular approach to the development of client-side rendered web applications. In this architecture, all essential resources are typically downloaded during the initial page load. As users interact with the application, the DOM is dynamically updated via JavaScript and HTTP requests to fetch necessary data from the server, completely eliminating the need for full page reloads. In addition, routing is fully managed on the client side, meaning that each time a user requests a view change, the URL is rewritten in a meaningful way to enhance the user experience. SPAs also allow developers to decide how to split the application logic between the server and the client. For example, a 'fat client' and a 'thin server' architecture implies that most of the logic is performed on the client side in the browser, while the server side is used primarily as a persistence layer. Alternatively, a 'thin client' and a 'fat server' approach implies that most of the logic is executed on the server, with the client simply displaying the content in a meaningful way. There are numerous frameworks, libraries, and tools available for building SPAs, the most popular being Angular, React, and Vue. These frameworks abstract DOM manipulation and provide lifecycle methods.\\

\noindent
The key advantage of SPAs is their dynamic, app-like user experience, as there are no full page reloads, simulating the experience of native applications for mobile devices or desktops. This allows users to seamlessly navigate between different parts of the application without long waiting times. Additionally, SPAs place a much lower load on the server.\\

\noindent
On the other hand, the obvious disadvantage is the longer initial load time because the entire application is downloaded upfront, instead of only the resources needed for the initial view. However, this can be mitigated with techniques such as code splitting or lazy loading. Another disadvantage is organizational: SPAs can quickly grow large, and if they are managed by different, especially distributed, teams working on the same codebase, different areas of the application may end up with inconsistent approaches and decisions. This can result in communication overhead between teams. Furthermore, as the codebase grows, team productivity may drastically slow down. \\

\noindent
SPAs are best suited for highly responsive and interactive applications with smaller data volumes, such as social networks or SaaS platforms. \cite{MezzaliraBuildingMf}\cite{Peltonen}