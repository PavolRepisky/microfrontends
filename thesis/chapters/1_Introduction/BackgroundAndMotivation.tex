\section{Background and Motivation}
The landscape of web development is constantly changing. We've progressed from simple, static text-based pages to highly dynamic interfaces, largely thanks to CSS, JavaScript, and AJAX. Eventually leading to adoption of Single Page Applications (SPAs), powered by frontend libraries and frameworks like React and Angular, reducing the gap between web and desktop applications \cite{WebDevFuture}\cite{MicrofrontendsStudy}. \\\\
The backend side of web apllications has also transformed. As applications expand, their codebases grow larger, often leading to excessive coupling and a lack of comprehensive understanding of how the application functions. This complexity causes challenges related to maintenance and scalability. This resulted to a shift from traditional monolithic structures to modular microservices-based architectures. In this architecture style, applications are aplit into smaller, independent services that can be developed and deployed separately, which simplifies the maintainability and scalability aspects. The microservices architecture has gained popularity in recent years and has been embraced by major companies such as Amazon and Netflix \cite{Microservices}. However, while this approach has addressed the backend struggles, similar issues are now being encountered on the frontend side. \\\\
The term "Micro Frontends" was first introduced in ThoughtWorks Technology Radar \cite{TechnologyRadar} in 2016. It can be described as an extension of microservices to the frontend layer. In a Microfrontend architecture, the application is divided into multiple features, each owned by independent teams. These features are then seamlessly composed together to form a whole \cite{MicrofrontendsInAction}\cite{MFApplication}. \\\\
While the microservices architecture has been extensivly studied, understood and adopted, its counterpart on the frontend side remains under-explored and under-theorized. This study tries to reduce this gap by conducting an analysis and implementation of microfrontends within the context of modern enterprise-level web development.